\documentclass{amsart}
\input{decls}
\title{Introduction to Homotopy Type Theory}
\author{}
\date{\today}
% \thanks{}
\begin{document}
\maketitle
\tableofcontents

\newcommand{\Bool}[0]{\ensuremath{\mathsf{Bool}}}
\newcommand{\Nat}[0]{\ensuremath{\mathbb{N}}}

\section{Introduction}
\label{sec:introduction}

\section{Martin-L\"{o}f Type Theory}
\label{sec:martin-lof-type-theory}
Martin-L\"{o}f Type Theory (MLTT) is similar to the Calculus of Inductive Construction (CIC) but without the universe Prop and has $\Sigma$ types.

\subsection{Boolean Type}
\label{sec:boolean-type}

\subsection{Natural Numbers}
\label{sec:natural-numbers}


\subsection{Exercises}
\label{sec:exercises}

\begin{ex}
  Define a function $\iota : \Bool \to \Nat$.
\end{ex}

\begin{ex}
  Define the addition function $+ : \Nat \to \Nat \to \Nat$ satisfying the following definitional equalities.
\end{ex}

\begin{ex}
  Define the multiplication function $\cdot : \Nat \to \Nat \to \Nat$ satisfying the following definitional equalities.
\end{ex}

\bibliographystyle{alpha}
\bibliography{all}

\end{document}
