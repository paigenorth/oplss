\documentclass{amsart}
\usepackage[utf8]{inputenc}
\usepackage{enumitem}
\usepackage{physics}
\usepackage{bm}
\usepackage{bbm}
\usepackage{hyperref}
\usepackage{tikz} 
\usetikzlibrary{positioning, arrows, matrix, intersections}
\usepackage{tabularx}
\newcolumntype{C}{>{\centering\arraybackslash}X}
\usepackage{xfrac}
\usepackage{adjustbox}
\usepackage{mathpartir}
\usepackage{quiver}
\usepackage{multicol}
\usepackage{amsmath}
\usepackage{pifont}
\newcommand{\cmark}{\ding{51}}%
\newcommand{\xmark}{\ding{55}}

\tikzset{snake it/.style={decorate, decoration=snake}}

\usepackage{minted}
\usemintedstyle{tango}

\theoremstyle{definition}
\newtheorem{eg}{Example}[section]
\newtheorem{ex}{Exercise}[section]
\newtheorem*{sol}{Solution}
\newtheorem*{warn}{Warning}
\newtheorem*{fact}{Fact}

\usepackage{pgfplots}
\pgfplotsset{compat=1.12}
\usepgfplotslibrary{fillbetween}

\title[Introduction to HoTT Notes]{OPLSS 2023 \\
Introduction to HoTT Notes}
\author{Pavel Kovalev, Sean O'Connor, Cassia Torczon, Frank Tsai}
\date{July 2023}


\newcommand{\N}{\mathbbm{N}}
\newcommand{\Z}{\mathbbm{Z}}
\newcommand{\Q}{\mathbbm{Q}}
\newcommand{\R}{\mathbbm{R}}
\newcommand{\ctx}{\ensuremath{\mathsf{~ctx}}}
\newcommand{\type}{\ensuremath{\mathsf{~type}}}
\newcommand{\defeq}{\ensuremath{\overset{\boldsymbol{\cdot}}{=}}}
\newcommand{\Unit}{\ensuremath{\mathsf{1}}}
\newcommand{\Bool}{\ensuremath{\mathsf{Bool}}}
\newcommand{\Prop}{\ensuremath{\mathsf{Prop}}}
\newcommand{\isProp}{\ensuremath{\mathsf{isProp}}}
\newcommand{\Set}{\ensuremath{\mathsf{Set}}}
\newcommand{\isSet}{\ensuremath{\mathsf{isSet}}}
\newcommand{\isContr}{\ensuremath{\mathsf{isContr}}}
\newcommand{\hLevel}[2]{\ensuremath{\mathsf{hLevel}~#1~#2}}
\newcommand{\isEquiv}{\ensuremath{\mathsf{isEquiv}}}
\newcommand{\idToEquiv}{\ensuremath{\mathsf{idToEquiv}}}
\newcommand{\Grp}{\ensuremath{\mathsf{Grp}}}
\newcommand{\W}{\ensuremath{\mathsf{W}}}
\newcommand{\U}{\ensuremath{\mathcal{U}}}
\newcommand{\True}{\ensuremath{\mathsf{true}}}
\newcommand{\False}{\ensuremath{\mathsf{false}}}
\newcommand{\Ind}{\ensuremath{\mathsf{ind}}}
\newcommand{\Hole}[1]{\fbox{?#1}}
\renewcommand{\emph}{\textbf}

\newcommand{\newcommenter}[3]{%
  \newcommand{#1}[1]{%
    \textcolor{#2}{\small\textsf{[{#3}: {##1}]}}%
  }%
}
\newcommenter{\FT}{red}{FT}

\usepackage[normalem]{ulem}
\newcommand{\surprising}{\textcolor{blue}{s\uwave{urprisin}g} $\text{}$}

\begin{document}

\maketitle
\tableofcontents

\section{Motivation}
\label{sec:motivation}

The fields of Type theory, Set theory, Topos theory, Category theory, Homotopy theory, and Functional programming all matured in the 20th century as their own distinct and influential fields of study. But near the beginning of the 21st century, more and more of the connections across these fields were recognized, which years later culminated in the incorporation of the field of Homotopy Type Theory (HoTT).

\begin{figure}[h]
    \centering
    \[\begin{tikzcd}
	{\text{Type theory}} & {\text{Logic}} & {\text{Set theory}} \\
	{\text{Functional programming}} & {\text{HoTT}} & {\text{Topos theory}} \\
	{\text{Homotopy theory}} && {\text{Category theory}}
	\arrow[no head, from=3-1, to=2-2]
	\arrow[no head, from=2-1, to=2-2]
	\arrow[no head, from=1-1, to=2-2]
	\arrow[no head, from=1-2, to=2-2]
	\arrow[no head, from=1-3, to=2-2]
	\arrow[no head, from=2-3, to=2-2]
	\arrow[no head, from=3-3, to=2-2]
	\arrow[no head, from=1-1, to=1-2]
	\arrow[no head, from=1-2, to=1-3]
	\arrow[no head, from=1-3, to=2-3]
	\arrow[no head, from=2-3, to=3-3]
	\arrow[no head, from=3-3, to=3-1]
	\arrow[no head, from=1-1, to=2-1]
	\arrow[color={rgb,255:red,0;green,0;blue,255}, shorten <=0pt, squiggly, no head, from=3-1, to=2-1]
\end{tikzcd}\]
    \caption{Connections across different fields of mathematics and computer science.
    The blue squiggly line is the surprising connection between homotopy theory and functional programming.}
    \label{fig:connections-across-fields}
\end{figure}

HoTT has strong ties to each of these other fields, and provides a fertile ground for investigating and proving \surprising connections between each of them. This lecture series will provide an introduction to this growing field and will help illuminate its relation to these other areas of study.
HoTT is built on top of Martin-L\"{o}f Type Theory (MLTT), which we introduce in \S\ref{sec:martin-lof-type-theory}.


\section{Martin-L\"{o}f Type Theory}
\label{sec:martin-lof-type-theory}
MLTT consists of 3 basic judgments.
An in-depth treatment of judgments can be found in \cite{ml:justif-log}.
The first judgment expresses that $\Gamma$ is a valid context.
\[
\Gamma \ctx
\]
The second judgment expresses that $\tau$ is a type under the context $\Gamma$, presupposing that $\Gamma \ctx$.
\[
\Gamma \vdash \tau \type
\]
Finally, the last judgment expresses that $t$ is an element of type $\tau$ under the context $\Gamma$, presupposing that $\Gamma \ctx$ and that $\Gamma \vdash \tau \type$.
\[
\Gamma \vdash t : \tau
\]
These three judgments have their corresponding equality judgments.
The first judgment expresses that two valid contexts, $\Gamma$ and $\Gamma'$, are definitionally equal.
\[
\Gamma \defeq \Gamma' \ctx
\]
The second judgment expresses that two types are definitionally equal under the same context $\Gamma$.
\[
\Gamma \vdash \tau \defeq \tau' \type
\]
And the last judgment expresses that two terms, $t$ and $t'$, of type $\tau$ under the context $\Gamma$ are definitionally equal.
\[
\Gamma \vdash t \defeq t' : \tau
\]
\begin{eg}[Some example types in MLTT]
\hfill
\begin{multicols}{2}
\begin{enumerate}
\item Empty type $\varnothing$.
\item Unit type $\Unit$.
\item The type of Boolean $\Bool$.
\item The type of natural numbers $\N$.
\item The types of equalities $=$.
\item Dependent product types $\Sigma$.
\item Dependent function types $\Pi$.
\item Universes $\U_{i}$.
\end{enumerate}
\end{multicols}
\vspace{-\baselineskip}
\begin{enumerate}
    \item[(5)] The types of well-founded trees $\W$. 
\end{enumerate}
\end{eg}

MLTT can be extended with additional features.
Two possible extensions are summarized in Table \ref{tb:extensions-of-mltt}.
\begin{table}[h]
    \centering
    \begin{tabular}{|c||c|c|}\hline
        Type Theory & Univalence Axiom & Higher Inductive Types\\\hline
        UTT  & \cmark & \xmark\\\hline
        HoTT & \cmark & \cmark\\\hline
    \end{tabular}
    \caption{Univalent Type Theory (UTT) is MLTT extended with the Univalence Axiom, and HoTT is UTT extended with higher inductive types.}
    \label{tb:extensions-of-mltt}
\end{table}

\subsection{Inductive Types}
\label{sec:inductive-types}
An inductive type is freely generated by its canonical elements.
To define an inductive type in Coq, one can write
\begin{figure}[H]
    \centering
    \begin{minted}{coq}
        Inductive T : Type := foo : T | bar : T.
    \end{minted}
\end{figure}
The type \texttt{T} is freely generated by its two canonical elements, namely \texttt{foo} and \texttt{bar}.
Coq generates an elimination rule for \texttt{T} automatically.

In general, to specify a new type in type theory we specify:
\begin{enumerate}
    \item how to form new types of this kind via \emph{formation rules}.
    For example, if $A$ and $B$ are types then $A \to B$ is a type.
    \item how to construct canonical elements of that type via \emph{introduction rules}.
    For example, a function type has one introduction rule, namely $\lambda$-abstraction.
    \item how to use elements of that type via \emph{elimination rules}.
    For example, a function type has one elimination rule, namely function application.
    \item how elimination rules act on introduction rules.
    For example, applying a $\lambda$-abstraction $(\lambda x.e)$ to a term $e'$ is definitionally equal to substituting $e'$ for $x$ in $e$, namely $e[\sfrac{e'}{x}]$. 
\end{enumerate}
See the HoTT book \cite{hottbook}.

\subsection{The Booleans}
\label{sec:the-booleans}

The type of Booleans is introduced using formation, introduction, elimination, and computation rules. 

\begin{itemize}
\item Formation rule:

\begin{mathpar}
\inferrule*[right=Bool-Form]{ }{\Gamma \vdash \Bool \type}
\end{mathpar}

\item Introduction rules:

\begin{mathpar}
\inferrule*[Right=Bool-Intros-True]{ }{\Gamma \vdash \True : \Bool}\and

\inferrule*[Right=Bool-Intros-False]{ }{\Gamma \vdash \False : \Bool}
\end{mathpar}

\item Elimination rule:

\begin{mathpar}
\inferrule*[Right=Bool-Elim]{\Gamma, b : \Bool \vdash D(b) \type \\ \Gamma \vdash f:D(\False) \\ \Gamma \vdash t:D(\True)}{\Gamma, b:\Bool \vdash \Ind_{\Bool}(f,t,b):D(b)}
\end{mathpar}

\item Computation rules:

\begin{mathpar}
\inferrule*[Right=Bool-Comp-1]
{\Gamma, b: \Bool \vdash D(b) \type \\ \Gamma \vdash f:D(\False) \\ \Gamma \vdash t:D(\True)}
{\Gamma \vdash \Ind_{\Bool}(f,t,\False) \defeq f : D(\False)}\and

\inferrule*[Right=Bool-Comp-2]
{\Gamma, b: \Bool \vdash D(b) \type \\ \Gamma \vdash f:D(\False) \\ \Gamma \vdash t:D(\True)}
{\Gamma \vdash \Ind_{\Bool}(f,t,\True) \defeq t : D(\True)}
\end{mathpar}

\end{itemize}

\begin{ex}\label{ex:not}
    Define a function $\mathsf{not}: \Bool \to \Bool$.
    Note that $\to$ works as usual.
\end{ex}

\subsection{The Natural Numbers}
\label{sec:the-natural-numbers}
The type of natural numbers is presented using formation, introduction, elimination, and computation rules. 

\begin{itemize}
\item Formation rule:

\begin{mathpar}
\inferrule*[Right=$\N$-form]
{ }
{\Gamma \vdash \mathbbm{N} \type}
\end{mathpar}

\item Introduction rules:

\begin{mathpar}
\inferrule*[Right=$\N$-intro-0]
{ }
{\Gamma \vdash 0 : \N}
\and
\inferrule*[Right=$\N$-intro-S]
{\Gamma \vdash n : \N}
{\Gamma \vdash S n : \N}
\end{mathpar}

\item Elimination rule:
\begin{mathpar}
\inferrule*[Right=$\N$-elim]
{\Gamma , n : \N \vdash D(n) \type \\\\ \Gamma \vdash z : D(0) \\\\ \Gamma , n : \N , h : D(n) \vdash i : D(S n)}
{\Gamma , n : \N \vdash \Ind_{\N}(z,i,n) : D(n)}
\end{mathpar}

\item Computation rules:

\begin{mathpar}
\inferrule*[Right=$\N$-comp-0]
{\Gamma , n : \N \vdash D(n) \type \\\\ \Gamma \vdash z : D(0) \\\\ \Gamma , n : \N , h : D(n) \vdash i : D(S n)}
{\Gamma \vdash \Ind_{\N}(z,i,0) \defeq z : D(0)}
\and
\inferrule*[Right=$\N$-comp-S]
{\Gamma , n : \N \vdash D(n) \type \\\\ \Gamma \vdash z : D(0) \\\\ \Gamma , n : \N , h : D(n) \vdash i : D(S n)}
{\Gamma, n : \N \vdash \Ind_{\N}(z,i,S n) \defeq i[\sfrac{\Ind_{\N}(z,i,n)}{h}] : D(S n)}
\end{mathpar}

\end{itemize}

\begin{ex}\label{ex:iota}
    Define $\iota : \Bool \to \N$.
\end{ex}

\begin{ex}\label{ex:add}
    Define $\mathsf{add} : \N \to \N \to \N$.
\end{ex}

\begin{ex}\label{ex:mult}
    Define $\mathsf{mult} :  \N \to \N \to \N$.
\end{ex}

\subsection{$\Sigma$-Types}
\label{sec:sigma-types}

Given $b : B \vdash E (b) \type$
(in Coq: \mintinline{coq}{E (b : B) : UU}),
we want to form a type whose terms are dependent pairs $\langle b, e \rangle$ where $b : B$, $e : E(b)$. We present this type using formation, introduction, elimination, and computation rules.

\begin{itemize}
\item Formation rule:

\begin{mathpar}
\inferrule*[Right=$\Sigma$-Form]
{\Gamma, b : B \vdash E(b) \type}
{\Gamma \vdash \Sigma_{x : B} E(x) \type}
\end{mathpar}

\item Introduction rule:

\begin{mathpar}
\inferrule*[Right=$\Sigma$-Intro]
{\Gamma \vdash b : B \\\\ \Gamma \vdash e : E(b)}
{\Gamma \vdash \langle b, e \rangle : \Sigma_{x : B} E(x)}
\end{mathpar}


\item Elimination rule:
\begin{mathpar}
\inferrule*[Right=$\Sigma$-Elim]
{\Gamma , z : \Sigma_{x : B} E(x) \vdash D(z) \type \\\\ 
 \Gamma, x : B, y : E(x) \vdash d(x,y) : D(\langle x, y \rangle)}
{\Gamma , z : \Sigma_{x : B} E(x) \vdash \Ind_{\Sigma,d}(z) : D(z)}
\end{mathpar}

\item Computation rule:

\begin{mathpar}
\inferrule*[Right=$\Sigma$-Comp]
{\Gamma , z : \Sigma_{x : B} E(x) \vdash D(z) \type \\\\ 
 \Gamma, x : B, y : E(x) \vdash d(x,y) : D(\langle x, y \rangle)}
{\Gamma , x : B, y : E(x) \vdash \Ind_{\Sigma,d}(\langle x, y \rangle) \defeq d(x, y) : D(\langle x, y \rangle)}
\end{mathpar}
\end{itemize}

We can think of $\Sigma_{x : B}E(x)$ as the disjoint union of sets indexed by the elements of $B$.
See Figure \ref{fig:sigma-type-as-total-space}.
\begin{figure}[h]
    \centering
    \begin{tikzpicture}
\node at (4,0) {$B$};
\node at (5,5) {$\Sigma_{x : B}E(x)$};

\node at (-1.2,-0.4) {$a$};
\draw[black,fill=black] (-1.5,-0.3) circle (.5ex);

\node at (0.1,0.2) {$b$};
\draw[black,fill=black] (-0.2,0.3) circle (.5ex);

\node at (2,-0.1) {$c$};
\draw[black,fill=black] (2,0.2) circle (.5ex);

\draw (0,5) ellipse (4cm and 2cm);
\draw (0,0) ellipse (3cm and 1cm);

\node at (-3,4.5) {$E(a)$};
\draw (-2,4.5) circle (0.5cm);

\node at (-1,5.5) {$E(b)$};
\draw (0,5.5) circle (0.5cm);

\node at (1,4.3) {$E(c)$};
\draw (2,4.3) circle (0.5cm);

\node at (0.4,2) {$\pi_{1}$};
\draw[->] (0,2.5) -- (0,1.5);
\end{tikzpicture}
    \caption{A graphical illustration of a $\Sigma$-type over some type $B$.}
    \label{fig:sigma-type-as-total-space}
\end{figure}

\begin{ex}
Construct a function $\pi_1 : \Sigma_{x : B} E(x) \to B$.
\end{ex}

\begin{ex}
Construct a function $\pi_2 : \Pi_{s : \Sigma_{x : B} E(x)} E(\pi_1 s)$. This is a dependently typed function that takes a term $s : \Sigma_{x : B} E(x)$ and returns something of type $E(\pi_1 s)$.
\end{ex}

\section{Digression: Pi Types}
Pi-types can be thought of as dependent function types. Note that they are \textbf{not} an inductive type, but we include a sketch of the rules for them here for clarity.

\begin{itemize}
\item Formation rule:

\begin{mathpar}
\inferrule*[Right=$\Pi$-Form]
{\Gamma, b : B \vdash E(b) \type}
{\Gamma \vdash \Pi{x : B} E(x) \type}
\end{mathpar}

\item Introduction rule:

\begin{mathpar}
\inferrule*[Right=$\Pi$-Intro]
{\Gamma, b : B \vdash e : E(b)}
{\Gamma \vdash \lambda b . e : \Pi{x : B} E(x)}
\end{mathpar}


\item Elimination rule:
\begin{mathpar}
\inferrule*[Right=$\Pi$-Elim]
{\Gamma \vdash f : \Pi{x : B} E(x) \\\\ \Gamma \vdash b : B}
{\Gamma \vdash f b : E(b)}
\end{mathpar}

\item $\beta$-equality

\begin{mathpar}
\inferrule*[Right=$\beta$-Eq]
{\Gamma , x : B \vdash e : E(x) \\\\ \Gamma \vdash b : B}
{\Gamma \vdash (\lambda x . e) b \defeq e[\sfrac{b}{x}] : E(b)}
\end{mathpar}

\item $\eta$-equality

\begin{mathpar}
\inferrule*[Right=$\eta$-Eq]
{\Gamma \vdash f : \Pi_{x : B} E(x)}
{\Gamma \vdash \lambda x . f x \defeq f : \Pi_{x : B} E(x)}
\end{mathpar}
\end{itemize}

\section{Types as logic, sets, and programs}
This section compares previously presented terms in light of the Curry-Howard correspondence, which relates logic to programs, and the Brouwer–Heyting–Kolmogorov interpretation, which provides an explanation of constructive logic. 
Table \ref{tb:types-as-logic-sets-programs} summarizes the interpretation of type theory in logic, sets, and programs.

\begin{table}[h]
    \centering
    \begin{tabular}{|c||p{0.2\textwidth}|p{0.2\textwidth}|p{0.3\textwidth}|}\hline
        Type Theory & Logic & Sets & Programs \\\hline
        $\Gamma \ctx$ & hypotheses & indexing set & names in scope \\\hline
        $\Gamma \vdash T \type$ & predicate $T$ on $\Gamma$ & family $T$ of sets indexed by $\Gamma$ & program specification using values from $\Gamma$\\\hline
        $\Gamma \vdash t : T$ & proof of $T$ & elements & program $t$ meeting the specification $T$\\\hline
        $\N$ & - & $\N$ & collection of programs with no input that output a natural number \\\hline
        $S + T$  $(\Sigma_{i : \Bool} T_i)$ & $\vee$ (disjunction) & $\sqcup$ (disjoint union) & $\vee$ over specifications \\\hline
        $S \times T$  $(\Sigma_{s : S} T)$ & $\wedge$ (conjunction) & $\times$ (Cartesian product) & $\wedge$ over specifications \\\hline
        $S \rightarrow T$  $(\Pi_{s : S} T)$ & $\Rightarrow$ (implication) & $T^S$ (exponential) & turns a program of type $S$ into a program of type $T$ \\\hline
        $\Sigma_{b : B} E(b)$ & $\exists b:B, E(b)$ (constructive existential) & $\sqcup_{b : B}E(b)$ & specification for producing an element $b : B$ meeting specification $E(b)$ \\\hline
        $\Pi_{b : B} E(b)$ & $\forall b : B, E(b)$ & $\Pi$ (set of sections) & specification for taking any program $b : B$ and outputting a program matching specification $E(b)$ \\\hline
    \end{tabular}
    \caption{Type theory interpreted in logic, sets, and programs.}
    \label{tb:types-as-logic-sets-programs}
\end{table}

\begin{eg}
    Given $x : \Bool \vdash \N \type$, the type $\Sigma_{x : \Bool} \N$ corresponds to the set $\N \sqcup \N$, and $\Pi_{x : \Bool} \N$ corresponds to the set $\N \times \N$ because there is no dependency between $x$ and $\N$ in the $\Pi$ type.
\end{eg}

\section{Identity Type}
\label{sec:identity}
Observe that $\mathsf{add}~0~m \defeq m : \N$ doesn't hold judgmentally because we don't know whether $m$ is 0 or is the successor of some natural number.
To ``prove" this equality, we need to use the elimination rule for $\N$.
However, since $\defeq$ is not a type, it does not make sense to make the judgment $\vdash t : \mathsf{add}~0~m \defeq 0$.
We can internalize the existing notion of equalities, namely judgmental equalities ($\defeq$).
This yields the identity types, giving us another notion of equalities, \emph{propositional equalities}.

Type formers often internalize existing concepts, i.e., turn them into types. For example, $\Bool$, $\N$, $\varnothing$, $\mathbbm{1}$, internalize existing notions of those things; $\Sigma$-types internalize context extensions (Figure \ref{fig:internalize}); $\Pi$-types internalize dependent types; and the universe type $\U$ internalizes the notion of something being a type. We will see how the identity type internalizes judgmental equality.

\begin{figure}[h]
    \centering
    \begin{tikzpicture}
    \node (0) at (0,0) {$x : B \vdash E(x) \type$};
    \node (1) at (-4.5,-3) {$x : B, y : E(x) \ctx$};
    \node (2) at (4.5, -3) {$\vdash \Sigma_{x : B}E(x) \type$};
    \path[->] 
    (0) 
    edge 
    node[sloped, anchor=center, above] {Context extension} (1);
    \path[->] 
    (0) 
    edge 
    node[sloped, anchor=center, above, text width=2cm] {$\Sigma$-formation} (2);
    \path[->] 
    (1) 
    edge[thick, dotted]
    node[sloped, anchor=center, above, text width=1.5cm] {Internalization} (2);
\end{tikzpicture}
    \caption{$\Sigma$-type internalizes context extension.}
    \label{fig:internalize}
\end{figure}

We start by presenting the formation, introduction, elimination, and computation rules for the identity type.

\begin{itemize}
\item Formation rule:

\begin{mathpar}
\inferrule*[Right=$\mathrm{=}$-Form]
{\Gamma \vdash A \type \\\\
 \Gamma \vdash a : A \\
 \Gamma \vdash b : A}
{\Gamma \vdash a =_{A} b \type}
\end{mathpar}

\item Introduction rule:

\begin{mathpar}
\inferrule*[Right=$\mathrm{=}$-Intro]
{\Gamma \vdash a : A}
{\Gamma \vdash r_{a} : a =_{A} a}
\end{mathpar}

\item Elimination rule:
\begin{mathpar}
\inferrule*[Right=$\mathrm{=}$-Elim]
{\Gamma , x : A , y : A, z : x =_A y \vdash D(x,y,z) \type \\\\
\Gamma, x : A \vdash d(x) : D(x,x,r_x)}
{\Gamma, x : A, y : A, z : x =_A y \vdash \Ind_{=}(d,x,y,z) : D(x, y, z)}
\end{mathpar}

\item Computation rule:

\begin{mathpar}
\inferrule*[Right=$\mathrm{=}$-Comp]
{\Gamma , x : A , y : A, z : x =_A y \vdash D(x,y,z) \type \\\\
\Gamma, x : A \vdash d(x) : D(x,x,r_x)}
{\Gamma, x : A \vdash \Ind_{=}(d,x,x,r_x) \defeq d(x) : D(x,x,r_x)}
\end{mathpar}
\end{itemize}


We talk about judgemental equalities (e.g., $a \defeq b : A$) at a ``meta" level.
The identity types (e.g., $r_a : a =_{A} b$) internalize these into the ``type-and-term" level.

The congruence rules for judgmental equalities say that
\begin{mathpar}
    \inferrule*
    {a : A, b : A \vdash a \defeq b : A \\ a : A \vdash a =_{A} a \type}
    {a : A, b : A \vdash (a =_{A} a) \defeq (a =_{A} b) \type}
\end{mathpar}
and that
\begin{mathpar}
    \inferrule*
    {a : A \vdash r_{a} : a =_{A} a \\ a : A, b : A \vdash (a =_{A} a) \defeq (a =_{A} b) \type}
    {a : A, b : A \vdash r_{a} : a =_{A} b}
\end{mathpar}
Thus, judgmentally equal terms are propositionally equal by reflexivity.

\begin{ex}
Show that $\mathsf{add}~0~n = n$ for all $n : \N$.
In other words, find a term that inhabits the type
\[
\prod_{n : \N} \mathsf{add}~0~n = n
\]
(Note: one of $\mathsf{add}~0~n) \defeq n$ and $\mathsf{add}~n~0 \defeq n$ will hold definitionally depending on how you defined $\mathsf{add}$.
Show that the other one holds judgmentally).
\end{ex}

\section{The Groupoidal Behavior of Types and the Space Interpretation}
The identity type is reflexive, symmetric, and transitive. Reflexivity comes by construction; for the other two, see Exercises \ref{ex:id-symmetry} and \ref{ex:id-transitivity}.
An identity type can also have multiple inhabitants, i.e., we can have $p, q : a =_A b$.
Since $p$ and $q$ are themselves terms, the equality formation rule allows us to form a new type.
\begin{mathpar}
    \inferrule*
    {\vdash p : a =_{A} b \\ \vdash q : a =_{A} b}
    {\vdash p =_{a =_{A} b} q \type}
\end{mathpar}

A type $A$ can be interpreted as a space, and inhabitants $p$ and $q$ of the identity type $a =_{A} b$ can be interpreted as two parallel \emph{paths} from $a$ to $b$ in the space $A$.
The inhabitants of the type $p =_{a =_{A} b} q$ can then be interpreted as \emph{2-dimensional} paths, or \emph{homotopies}, between the two 1-dimensional paths $p$ and $q$.
Similarly, we can form the type $r =_{p =_{(a =_{A} b)} q} s$ of 3-dimensional paths between two parallel 2-dimensional paths.
This process can be repeated ad infinitum, giving higher dimensional structures.
See Figure \ref{fig:n-paths}.

\begin{figure}[h]
    \centering
    % https://q.uiver.app/#q=WzAsMixbMCwwLCJhIl0sWzQsMCwiYiJdLFswLDEsIiIsMCx7ImN1cnZlIjotNX1dLFswLDEsIiIsMix7ImN1cnZlIjo1fV0sWzIsMywiIiwyLHsib2Zmc2V0IjoxLCJjdXJ2ZSI6Mywic2hvcnRlbiI6eyJzb3VyY2UiOjIwLCJ0YXJnZXQiOjIwfX1dLFsyLDMsIiIsMCx7Im9mZnNldCI6LTEsImN1cnZlIjotMywic2hvcnRlbiI6eyJzb3VyY2UiOjIwLCJ0YXJnZXQiOjIwfX1dLFs0LDUsIiIsMCx7InNob3J0ZW4iOnsic291cmNlIjoyMCwidGFyZ2V0IjoyMH0sImxldmVsIjoyfV0sWzQsNSwiIiwyLHsic2hvcnRlbiI6eyJzb3VyY2UiOjIwLCJ0YXJnZXQiOjIwfSwibGV2ZWwiOjEsInN0eWxlIjp7ImhlYWQiOnsibmFtZSI6Im5vbmUifX19XV0=
\[\begin{tikzcd}
	a &&&& b
	\arrow[""{name=0, anchor=center, inner sep=0}, curve={height=-30pt}, from=1-1, to=1-5]
	\arrow[""{name=1, anchor=center, inner sep=0}, curve={height=30pt}, from=1-1, to=1-5]
	\arrow[""{name=2, anchor=center, inner sep=0}, shift right=1, curve={height=18pt}, shorten <=10pt, shorten >=10pt, Rightarrow, from=0, to=1]
	\arrow[""{name=3, anchor=center, inner sep=0}, shift left=1, curve={height=-18pt}, shorten <=10pt, shorten >=10pt, Rightarrow, from=0, to=1]
	\arrow[shorten <=8pt, shorten >=8pt, Rightarrow, from=2, to=3]
	\arrow[shorten <=8pt, shorten >=8pt, no head, from=2, to=3]
\end{tikzcd}\]
    \caption{An illustration of higher dimensional paths.}
    \label{fig:n-paths}
\end{figure}

%Why is this "the first homotopical phenomenon"
%Make the diagram

\begin{ex}[Symmetry]\label{ex:id-symmetry}
    Define a function that produces the inverse of a path.
    \[
        (-)^{-1} : (a = b) \to (b = a)
    \]
\end{ex}

\begin{ex}[Transitivity]\label{ex:id-transitivity}
    Define a function that produces the composition of two paths.
    \[
        - \cdot - : (a = b) \to (b = c) \to (a = c)
    \]
\end{ex}

\begin{table}[h]
    \centering
    \begin{tabular}{|c|c|}\hline
        Equality & Homotopy \\\hline
        reflexivity & constant path \\\hline
        symmetry & inverse path \\\hline
        transitivity & path concatenation \\\hline
    \end{tabular}
    \caption{Exercises \ref{ex:id-symmetry} and \ref{ex:id-transitivity} reveal the connections between identity types and homotopies.}
    \label{tab:equalities-and-homotopy}
\end{table}

\begin{ex}
    Give an example of two equalities in some type that are equal to each other in the relevant identity type and prove that they are equal, i.e., pick a specific type $A$ and two terms $a$ and $b$, construct two distinct terms $p, p' : a =_A b$, and prove $\alpha : p \mathbf{=}_{a =_A b} p'$.
\end{ex}

\begin{ex}
    Show that functions $f : A \to B$ respect equalities.
    In other words, construct the following function:
    \[
        \mathsf{ap}_{f} : (a =_{A} b) \to (f(a) =_{B} f(b))
    \]
    The notation $\mathsf{ap}_{f}$ can be read as the \underline{a}ction on \underline{p}aths of $f$ \cite{hottbook}.
\end{ex}

%\begin{figure}[h]
%    \centering
%    \begin{tikzpicture}
\node at (4,0) {$B$};
\node at (5,5) {$\Sigma_{x : B}E(x)$};

\node at (-2,0) {$a$};
\draw[black,fill=black] (-1.5,0) circle (.5ex);

\node at (2.5,0) {$b$};
\draw[black,fill=black] (2,0) circle (.5ex);

\draw (0,5) ellipse (4cm and 1.3cm);
\draw (0,0) ellipse (3cm and 1cm);

\path[draw, snake it] (-1.5,0) -- (2,0);

\node at (0, 0.5) {$p$};

\node at (-2,5) {$f(a)$};
\draw[black,fill=black] (-1.5,5) circle (.5ex);

\node at (2.5,5) {$f(b)$};
\draw[black,fill=black] (2,5) circle (.5ex);

\path[draw, snake it] (-1.5,5) -- (2,5);
\node at (0, 5.5) {$\mathsf{ap}_{f}~p$};

\node at (0.4,2.2) {$\pi_{1}$};
\draw[->] (0,3) -- (0,1.5);
\end{tikzpicture}
%    \caption{Caption}
%    \label{fig:ap}
%\end{figure}

Veovosky showed that MLTT can be interpreted in a category of spaces (the category of Kan complexes).
\begin{table}[h]
    \centering
    \begin{tabular}{|c|c|}\hline
        MLTT & Space\\\hline
        types $T$ & spaces $T$\\\hline
        terms $t$ & points $t \in T$\\\hline
        dependent types $x : B \vdash E(x) \type$ & fibrations 
            $\begin{array}{c}
            E\\
            \downarrow\\
            B
            \end{array}$\\\hline
        equalities & paths/homotopies\\\hline
    \end{tabular}
    \caption{The space interpretation of MLTT.}
    \label{tab:the-space-interpretation-of-mltt}
\end{table}

\begin{ex}[Transport]
    Show that for any dependent type $x : B \vdash E(x) \type $, any terms $b, b' : B$, and any path $p : b =_B b'$, there is a function $p_{*} : E(b) \to E(b')$.
\end{ex}
This ensures that every dependent type we can construct respects propositional equality. 
If we think of $E$ as a predicate on $B$, this means that if $E(b)$ is true and $b =_B b'$, then $E(b')$ is true.
This is known as the \emph{indiscernibility of identicals}.

This is part of a more sophisticated relationship between type theory and homotopy theory (Quillen model category theory, or QMC theory). 
Transport says that $\pi_{1} : \Sigma_{b : B} E(b) \rightarrow B$ behaves like a fibration in a QMC. For more information, see Appendix $\ref{sec:transport-explanation}$.

\section{Equivalence}
For types $S, T$, there is a notion of equivalence $S \simeq T$. We will come back to this later, but it is similar to the notion $\Sigma_{f : S \rightarrow T}$ $\Sigma_{g : T \rightarrow S} (\Pi_{x : S} g(f (x)) = x) \times (\Pi_{y : T} f(g(y)) = y)$. (This is informally similar to idea of there being an isomorphism $f$ between $S$ and $T$, with inverse $g$.)

\section{Characterizing Equality in Standard Types}

\begin{itemize}
\item[\Bool:] We can show $\False = \False$,
$\True = \True$,
$\False \neq \True$. (Note that $\False \neq \True$ means that there is a function $\False \neq \True$ defined as $(\False = \True)\to \emptyset$. More generally, $\neg P$ is defined as $P\to\emptyset$.)
\item[$\N$:] We can show $S n = S m \Rightarrow n = m$ and $0 \neq S n$.
\item[$\Sigma$-types:] We can show that for all $s, t : \Sigma_{a : A} B(a)$, we have $(s \mathbf{=}_{\Sigma_{a : A} B(a)} t) \simeq \Sigma_{p : \pi_1 s = \pi_1 t} \mathbf{tr}_p \pi_2 s = \pi_2 t$.
\item[$\Pi$-types:] \textsc{(functional extensionality)} We might want that $\forall f, g : \Pi_{a : A} B(a)$, $(f = g) \simeq \Pi_{x : A} f(x) = g(x)$, but this is functional extensionality ($\mathbf{funext}$), and it is \textbf{not provable} in MLTT. This is validated by interpretations  in logic, sets, and spaces (i.e., functional extentionality is true in all of those systems).
\item[$=$-types:] \textsc{(uniqueness of identity proofs)} We might want that $\forall p, q : a = b$, $(p = q) \simeq \mathbbm{1}$, but this is uniqueness of identity proofs (UIP), which is \textbf{not provable} in MLTT. This is validated by interpretations in logic and sets, but not in spaces. 
\item[$U$-types:] \textsc{(univalence)} We might want that $\forall S, T : U$, $(S = T) \simeq (S \simeq T)$, but this is univalence ($\mathbf{UA}$), which is \textbf{not provable} in MLTT. This is validated by interpretation in \textbf{spaces}.
\end{itemize}

\begin{figure}[h]
    \centering
    % https://q.uiver.app/#q=WzAsMyxbMCwwLCJVQSJdLFs2LDAsIlVJUCArIEVSIl0sWzMsMywiRnVuRXh0Il0sWzAsMiwiaW1wbGllcyIsMl0sWzEsMiwiaW1wbGllcyJdLFswLDEsImluY29tcGF0aWJsZSIsMCx7InN0eWxlIjp7ImJvZHkiOnsibmFtZSI6InNxdWlnZ2x5In0sImhlYWQiOnsibmFtZSI6Im5vbmUifX19XV0=
\[\begin{tikzcd}
	\mathrm{UA} &&&&&& {\mathrm{UIP \& ER}} \\
	\\
	\\
	&&& \mathrm{FunExt}
	\arrow["\mathrm{Implies}"', from=1-1, to=4-4]
	\arrow["\mathrm{Implies}", from=1-7, to=4-4]
	\arrow["\mathrm{Incompatible}", squiggly, no head, from=1-1, to=1-7]
\end{tikzcd}\]
    \caption{Both UA and UIP with equality reflection (ER) imply functional extensionality.
    However, UA and UIP are incompatible.}
    \label{fig:ua-uip-funext}
\end{figure}

To avoid inconsistency, we cannot have both $\mathbf{UA}$ and $\mathbf{UIP}$. We choose $\mathbf{UA}$ (and so also get $\mathbf{funext}$). 
Both choices make sense, but choosing $\mathbf{UIP}$ would send us in the direction of set theory, not HoTT.

\section{Homotopy Levels}
In this section, we introduce \emph{homotopy levels}, or h-levels for short, which are defined recursively as follows:
\begin{align*}
    \mathsf{hLevel}~0~T &:= \sum_{t : T}\prod_{s : T}(s =_T t)\\
    \mathsf{hLevel}~S(n)~T &:= \prod_{s,t : T}\mathsf{hLevel}~n~(s = t)
\end{align*}

\subsection{h-level 0}\label{sec:h-level-0}
A type $T$ is \emph{contractible} (has h-level 0) if there is a \emph{center of contraction} $t$, and for every element $s$ of that type, there is a path from $s$ to $t$.
\[
    \isContr(T) := \sum_{t : T}\prod_{s : T} (s = t)
\]
\begin{ex}
Show that $\mathbbm{1}$ is contractible.
\end{ex}

\begin{ex}
Show that if $T$ is contractible and inhabited, then $T \simeq \mathbbm{1}$.
\end{ex}

%\begin{figure}[h]
%    \centering
%    \begin{tikzpicture}
    \draw (0,0) circle (3cm);
    \node (t) at (0,0) {};
    \draw[black, fill=black] (0, 0) circle (.5ex);

    \node (1) at (2,2) {};
    \node (2) at (-1.3,1) {};
    \node (3) at (-2.3,-0.3) {};
    \node (4) at (1.7,-2);

    \draw[->] (1) -- (t);
    \draw[->] (2) -- (t);
    \draw[->] (3) -- (t);
    \draw[->] (4) -- (t);
\end{tikzpicture}
%    \caption{Caption}
%    \label{fig:h-level0}
%\end{figure}

\subsection{h-level 1}\label{sec:h-level-1}
A type $T$ is a \emph{proposition} if it has h-level 1.
\[
    \hLevel{1}{T} := \prod_{s,t : T}\sum_{q : s = t}\prod_{p : s = t}(p = q)
\]
\begin{ex}
Show that $\emptyset$, $\mathbbm{1}$ are propositions.
\end{ex}

\begin{ex}
Show that any contractible type is a proposition.
\end{ex}

\begin{ex}
Show that if a proposition is inhabited, then it is contractible. (This says that a proposition is always informally equivalent to $\emptyset$ or $\mathbbm{1}$, i.e., truth values) %I don't think this is actually what this exercise shows but I'm not sure what the correct interpretation of this comment in the notes is
\end{ex}

\subsection{h-level 2}\label{sec:h-level-2}
A type $T$ is a \emph{set} if it has h-level 2, i.e., when the equalities are propositions.
\[
    \hLevel{2}{T} := \prod_{s,t : T}\prod_{p,q : s = t}\sum_{u : p = q}\prod_{v : p = q}(v = u)
\]
$\N$ and $\Bool$ are sets.
\subsection{h-level 3}\label{sec:h-level-3}
A type $T$ is a \emph{groupoid} when it has h-level 3, i.e., when the homotopies are propositions.
\[
    \hLevel{3}{T} := \prod_{s,t : T}\prod_{p,q : s = t}\prod_{u,v : p = q}\sum_{w : u = v}\prod_{x : u = v}(x = w)
\]
A universe $\U$ has h-level at least 3.

\begin{ex}
Show that if a type $T$ has h-level $n$, then it has h-level $n + 1$.
\end{ex}

\section{Equivalences}
Sometimes we want types to be propositions, i.e., to have no structure. In other situations we may want to have the structure.

Given two types $A$ and $B$ and a function $f : A \rightarrow B$, we want a \textbf{proposition} $\isEquiv(f)$ encapsulating the idea that $f$ is a bijection, or ``equivalence''.

The type $\Sigma_{g : B \rightarrow A} (f \circ g = 1_B) \times (g \circ f = 1_A)$ encapsulates this idea, but it is not a  proposition. Alternatively, we can define adjoint equivalence:

\begin{align*}
\isEquiv(f) &:= \Pi_{b : B} \isContr(\Sigma_{a : A} f(a)=b)
\end{align*}



The part $\isContr(\Sigma_{a : A} f(a)=b)$ states that 1 is a fiber. %?

\textsc{Definition.} Given two types $A$ and $B$, a function $f : A \rightarrow B$ is an \textbf{equivalence} if we have $\isEquiv(f)$.

We write $A \simeq B$ to denote $\Sigma_{f : A \rightarrow B} \isEquiv(f)$. For every type $A$, we have $A \simeq B$, so we can define a function $\idToEquiv: (A=B) \rightarrow (A \simeq B)$.

\textsc{Definition.} The \textbf{univalence axiom} asserts that we have a term $ua : \isEquiv(\idToEquiv)$.

\section{Univalence for Logic and Sets}

We define $\Prop := \Sigma_{P:\type} \isProp (P)$. (Note that since $\isProp$ is a proposition, $\isProp \subseteq \type$.)

\begin{fact}
The univalence axiom implies $(P =_{\Prop} Q)\simeq (P \leftrightarrow Q)$. 
\end{fact}

We can also define $\Set := \Sigma_{S:\type} \isSet (S)$.

\begin{fact}
The univalence axiom implies $(P =_{\Set} Q)\simeq (P \cong Q)$.
\end{fact}

Lastly, we can define $ \Grp := \Sigma_{G: \Set} \Sigma_{e:G} \Sigma_{m: G\to G \to G} \Sigma_{i:G\to G} \Pi_{x:G} (m(e,x)=x) \times (m(x,e)=x) \times \Pi_{x,y,z:G}((xy)z=x(yz)) \times \Pi_{x:G}(m(ix,x)=x \times (m(x,ix)=e))$


\begin{fact}
The univalence axiom implies $(P =_{\Grp} Q)\simeq (P \cong Q)$.    
\end{fact}


\bibliographystyle{alpha}
\bibliography{all}

\appendix

\section{Transport and Fibrations}
\label{sec:transport-explanation}

As mentioned earlier, the ability to transport along an equality corresponds to the projection map $\pi_1 : \Sigma_{b : B} E(b)$ behaving like a fibration in a QMC. We will now discuss one way to see this, taking the liberty to interpret equalities as a map from the interval type into the target type, as is possible in a QMC (as well as in cubical formulations of HoTT).

\begin{figure}[h]
    \centering
    % https://q.uiver.app/#q=WzAsNCxbMCwwLCIqIl0sWzIsMCwiXFxTaWdtYV97YiA6IEJ9IEUoYikiXSxbMCwyLCJJIl0sWzIsMiwiQiJdLFsyLDMsInAiLDJdLFswLDEsIlxcdGV4dHtjb25zdH1feyhiLCBlKX0iXSxbMCwyLCJcXHRleHR7Y29uc3R9X3swfSIsMl0sWzEsMywiXFxwaV8xIl0sWzIsMSwiaCIsMix7InN0eWxlIjp7ImJvZHkiOnsibmFtZSI6ImRhc2hlZCJ9fX1dXQ==
\[\begin{tikzcd}
	{*} && {\Sigma_{b : B} E(b)} \\
	\\
	I && B
	\arrow["p"', from=3-1, to=3-3]
	\arrow["{\text{const}_{(b, e)}}", from=1-1, to=1-3]
	\arrow["{\text{const}_{0}}"', from=1-1, to=3-1]
	\arrow["{\pi_1}", from=1-3, to=3-3]
	\arrow["h"', dashed, from=3-1, to=1-3]
\end{tikzcd}\]
    \caption{Given that the outer diagram commutes, we have that if $\text{pr}_1$ is a fibration, there exists a homotopy lifting map $h$ making everything commute.}
    \label{fig:transport-as-fibration}
\end{figure}

Consider Figure \ref{fig:transport-as-fibration}. If the outer diagram commutes, we have the data $b : B$, $e : E(b)$, and a path $p : b = b'$ for some (unstated) $b' : B$.

This follows because, when interpreting $p$ as a map from $I$ to $B$, evaluating $p$ at $0 : I$ must result in $b$ by commutativity.

Then, if $\pi_1$ is a fibration, we get that there exists a homotopy lift $h : (b, e) = (b', e')$ for some (unstated) $e' : E(b')$ that makes the diagram commute.

This is because, when interpreting $h$ as a map from $I$ to $\Sigma_{b : B} E(b)$, evaluating $p$ at $0 : I$ must result in $(b, e)$ by commutativity.

Furthermore, evaluating $h$ at $1 : I$ and then applying $\pi_1$ must result in the same value as evaluating $p$ at $1 : I$, also by commutativity.

Thus, given such an $h$, we can define $tr_p (e)$ to equal $\pi_2 (h (1)) : E(b')$. Thus, if $\pi_1$ is always a fibration, given a term $p : b = b'$, we can define $tr_p : E(b) \to E(b')$.

In the reverse direction, given a term $tr_p : E(b) \to E(b')$ along with the property that $tr_{r_b} \defeq \text{id}_{E(b)}$ (which holds when defining transport using path induction), given that the outer diagram in Figure \ref{fig:transport-as-fibration} commutes, we can define $h : (b, e) = (b', tr_p (e))$.

To do this, we can use the elimination rule for identity types to reduce this problem to the case where $b' \defeq b$ and $p \defeq r_b$. As such, we now want to define $h : (b, e) = (b', tr_{r_b} (e))$.

But as $(b', tr_{r_b} (e))$ definitionally reduces to $(b, \text{id}_{E(b)} (e))$ and then again to $(b, e)$, we get that $h : (b, e) = (b, e)$.

As such, we can just let $h$ be $r_{(b, e)}$ to finish the proof.

Thus, we have that if $\pi_1$ is a fibration, we can construct transport along a path $p : b = b'$ as $tr_b : E(b) \to E(b')$, and if we have transport we can show that $\pi_1$ is a fibration.

As a bonus exercise, prove that this map taking in $tr_p$ and output $h$ is an equivalence!

\section{Solutions}
\label{sec:exercise-solutions}
\begin{sol}[\ref{ex:not}]
    The function $\mathsf{not}$ has the following definitional equalities:
    \begin{enumerate}
        \item $\mathsf{not}~\False \defeq \True$
        \item $\mathsf{not}~\True \defeq \False$
    \end{enumerate}
    Let $b : \Bool$ be given and $b : \Bool \vdash \Bool \type$ be the motive of induction.
    By induction on $b$, it suffices to specify $\vdash \Hole{1} : \Bool$ and $\vdash \Hole{2} : \Bool$.
    \begin{mathpar}
        \inferrule*[right=Abs]
        {
            \inferrule*[Right=Bool-Elim]
            {b : \Bool \vdash \Bool \type \\ 
            \vdash \Hole{1} : \Bool \\ 
            \vdash \Hole{2} : \Bool}
            {b : \Bool \vdash \Ind_{\Bool,\Hole{1},\Hole{2}}(b) : \Bool}
        }
        {\vdash \lambda b : \Bool.~\Ind_{\Bool,\Hole{1},\Hole{2}}(b) : \Bool \to \Bool}
    \end{mathpar}
    To satisfy the first definitional equality, we choose $\Hole{1} := \True$.
    Similarly, to satisfy the second definitional equality, we choose $\Hole{2} := \False$.
    \begin{mathpar}
        \inferrule*[right=Abs]
        {
            \inferrule*[Right=Bool-Elim]
            {b : \Bool \vdash \Bool \type \\ 
            \vdash \True : \Bool \\ 
            \vdash \False : \Bool}
            {b : \Bool \vdash \Ind_{\Bool,\True,\False}(b) : \Bool}
        }
        {\vdash \lambda b : \Bool.~\Ind_{\Bool,\True,\False}(b) : \Bool \to \Bool}
    \end{mathpar}
    \begin{warn}
        The order of $\Hole{1}$ and $\Hole{2}$ is reversed in the Coq implementation.
    \end{warn}
\end{sol}

\begin{sol}[\ref{ex:iota}]
    The function $\iota$ has the following definitional equalities:
    \begin{enumerate}
        \item $\iota~\False \defeq 0$
        \item $\iota~\True \defeq 1$
    \end{enumerate}
    Let $b : \Bool$ be given and $b : \Bool \vdash \N \type$ be the motive of induction.
    By induction on $b$, it suffices to specify two terms, $\vdash \Hole{1} : \N$ and $\vdash \Hole{2} : \N$.
    \begin{mathpar}
        \inferrule*[right=Abs]
        {
            \inferrule*[Right=Bool-Elim]
            {b : \Bool \vdash \N \type \\\\ \vdash \Hole{1} : \N \\ \vdash \Hole{2} : \N}
            {b : \Bool \vdash \Ind_{\Bool,\Hole{1},\Hole{2}}(b) : \Bool}
        }
        {\vdash \lambda b : \Bool.~\Ind_{\Bool,\Hole{1},\Hole{2}}(b) : \Bool \to \N}
    \end{mathpar}
    To satisfy the definitional equalities, we choose $\Hole{1} := 0$ and $\Hole{2} := 1$.
    \begin{mathpar}
        \inferrule*[right=Abs]
        {
            \inferrule*[Right=Bool-Elim]
            {b : \Bool \vdash \N \type \\\\ \vdash 0 : \N \\ \vdash 1 : \N}
            {b : \Bool \vdash \Ind_{\Bool,0,1}(b) : \Bool}
        }
        {\vdash \lambda b : \Bool.~\Ind_{\Bool,0,1}(b) : \Bool \to \N}
    \end{mathpar}
    \begin{warn}
        The order of $\Hole{1}$ and $\Hole{2}$ is reversed in the Coq implementation.
    \end{warn}
\end{sol}

\begin{sol}[\ref{ex:add}]
    The function $\mathsf{add}$ has the following definitional equalities:
    \begin{enumerate}
        \item $\mathsf{add}~n~0 \defeq n$
        \item $\mathsf{add}~n~S(m) \defeq S(\mathsf{add}~n~m)$
    \end{enumerate}
    Let $n : \N$, $m : \N$ be given, and $n : \N, m : \N \vdash \N \type$ be the motive of induction.
    By induction on $m$, it suffices to specify $n : \N \vdash \Hole{1} : \N$ and $n : \N, m : \N, h : \N \vdash \Hole{2} : \N$.
    \begin{mathpar}
        \inferrule*[right=Abs]
        {
            \inferrule*[Right=$\N$-Elim]
            {n : \N, m : \N \vdash \N \type \\\\ 
             n : \N \vdash \Hole{1} : \N \\\\ 
             n : \N, m : \N, h : \N \vdash \Hole{2} : \N}
            {n : \N, m : \N \vdash \Ind_{\N,\Hole{1},\Hole{2}}(m) : \N}
        }
        {\vdash \lambda n : \N.~\lambda m : \N.~\Ind_{\N,\Hole{1},\Hole{2}}(m) : \N \to \N \to \N}
    \end{mathpar}
    To satisfy the first definitional equality, choose $\Hole{1} := n$.
    To satisfy the second definitional equality, choose $\Hole{2} := S(h)$
    \begin{mathpar}
        \inferrule*[right=Abs]
        {
            \inferrule*[Right=$\N$-Elim]
            {n : \N, m : \N \vdash \N \type \\\\ 
             n : \N \vdash n : \N \\\\ 
             n : \N, m : \N, h : \N \vdash S(h) : \N}
            {n : \N, m : \N \vdash \Ind_{\N,n,S(h)}(m) : \N}
        }
        {\vdash \lambda n : \N.~\lambda m : \N.~\Ind_{\N,n,S(h)}(m) : \N \to \N \to \N}
    \end{mathpar}
\end{sol}

\begin{sol}[\ref{ex:mult}]
    The function $\mathsf{mult}$ has the following definitional equalities:
    \begin{enumerate}
        \item $\mathsf{mult}~n~0 \defeq 0$
        \item $\mathsf{mult}~n~S(m) \defeq \mathsf{add}~n~(\mathsf{mult}~n~m)$
    \end{enumerate}
    Let $n : \N$, $m : \N$ be given, and $n : \N, m : \N \vdash \N \type$ be the motive of induction.
    By induction on $m$, it suffices to specify $n : \N \vdash \Hole{1} : \N$ and $n : \N, m : \N, h : \N \vdash \Hole{2} : \N$.
    \begin{mathpar}
        \inferrule*[right=Abs]
        {
            \inferrule*[Right=$\N$-Elim]
            {n : \N, m : \N \vdash \N \type \\\\ 
             n : \N \vdash \Hole{1} : \N \\\\ 
             n : \N, m : \N, h : \N \vdash \Hole{2} : \N}
            {n : \N, m : \N \vdash \Ind_{\N,\Hole{1},\Hole{2}}(m) : \N}
        }
        {\vdash \lambda n : \N.~\lambda m : \N.~\Ind_{\N,\Hole{1},\Hole{2}}(m) : \N \to \N \to \N}
    \end{mathpar}
    To satisfy the first definitional equality, choose $\Hole{1} := 0$.
    To satisfy the second definitional equality, choose $\Hole{2} := \mathsf{add}~n~h$.
    \begin{mathpar}
        \inferrule*[right=Abs]
        {
            \inferrule*[Right=$\N$-Elim]
            {n : \N, m : \N \vdash \N \type \\\\ 
             n : \N \vdash 0 : \N \\\\ 
             n : \N, m : \N, h : \N \vdash \mathsf{add}~n~h : \N}
            {n : \N, m : \N \vdash \Ind_{\N,0,\mathsf{add}~n~h}(m) : \N}
        }
        {\vdash \lambda n : \N.~\lambda m : \N.~\Ind_{\N,0,\mathsf{add}~n~h}(m) : \N \to \N \to \N}
    \end{mathpar}
\end{sol}

\section{Links and FAQ}

\begin{enumerate}
    \item \textbf{How does univalence imply functional extensionality?} Here is a blog post detailing that: \href{https://homotopytypetheory.org/2014/02/17/another-proof-that-univalence-implies-function-extensionality/}{Another proof that univalence implies functional extensionality by Dan Licata}.

    Perhaps the biggest takeaway from this article is that this proof that univalence implies functional extensionality is very closely related to the proof in cubical type theory of functional extensionality, which doesn't use univalence! There is something bigger at play here, demonstrating that univalence in some way makes things in MLTT "work better" than it used to.

    Other than that, I highly recommend checking out the article linked above, as it is by far the clearest explanation of this proof that I've seen.
    \item \textbf{Are h-levels different from universe levels?} Yes! In a type hierarchy such as the one used by Coq, each $\texttt{Type}_i$ has h-level $\infty$.
    \item \textbf{How does the inductive definition of identity types allow for non-canonical terms?} The crucial difference for identity types is that we are inductively defining a \textit{family} of types in \ref{sec:identity}. The types build by these definitions are indexed by the terms being equated, allowing us to have non-canonical terms.
\end{enumerate}

\end{document}
